\documentclass{article}
\usepackage[utf8]{inputenc}
\usepackage{fancyhdr}
\usepackage{graphicx}
\usepackage{geometry}
\usepackage{float}

% ---- Commands ------- %
\newcommand{\documentNumber}[1]{
    \LARGE  \textbf{ Kravprocessbok }
    \\
    \medskip
}
\newcommand{\documentVersion}[1]{
    \medskip
}
\newcommand{\documentTitle}[1]{
    \centerline{\rule{13cm}{0.4pt}}
    \bigskip \bigskip
    \LARGE \textbf{Projekt IDA3} \\
    \bigskip
    \LARGE {#1} \\
    \bigskip \bigskip
    \centerline{\rule{13cm}{0.4pt}}
}

\newcommand{\documentDate}[1]{
    \date {#1} 
}


\renewcommand{\arraystretch}{1.7}  % Vertical padding for tables

\renewcommand{\contentsname}{Innehållsförtäckning}

% --- Header & Footer ---- %
\pagestyle{fancy}
\lhead{\leftmark}
\rhead{}
\rfoot{\thepage}
\cfoot{}
\lfoot{}


% ------------------------------------------------ #

% ----- FILL THIS ----- %
\title {
    \documentNumber {01}    

    % Full name - SHORTNAME
    \documentTitle {Helsingborg Event and Convention Bureau}
    
    % Format: YYYY-MM-DD
    \documentDate {2021-08-12}
    
    \author{Anna Bergvall - Oscar Blixt - Pontus Persson - Filip Sjövall - David Vilppu}
}

\begin{document}
\addtocontents{toc}{\protect\setcounter{tocdepth}{2}}
\maketitle

\thispagestyle{empty}

\newpage

\tableofcontents


\newpage

\section{Loggbok}
    \subsection{Vecka 36}
    \\
    Denna veckan har varit stort fokus på att komma igång. Det var första veckan vi träffades som grupp och pratade ihop oss om vad projektet kommer innebära. Arbetet har därför legat mer åt att organisera projektgruppen än kravhantering. \\ Varsitt möte med Malin Planander på Miljlbron samt Malin Hollgren, våran "kund" och två övningar inom kursen gav oss däremot en skjuts med att komma igång med kravarbetet, så under veckan har vi samlat på oss en del krav som nu måste konkritiseras och dokumenteras. 
    
    \newpage
    

\section{Intressenter och elicitering v.36}

    \subsection{Nyckelintressenter}
        \begin{itemize}
            \item \textbf{Malin Hollberg, Helsingborg Convention and Event Bureau}.
                \\
            \item \textbf{Slutanvändarna}. Det vill säga de som jobbar inom supply chain.
                \\
            \item \textbf{GDSM} - Global Destination Sustainability Movment. Företaget som tar hand om statistiken vårt system kommer att sammanställa.
             \\
            \item \textbf{Malin Planander, Miljöbron}.
        \end{itemize}
        
    \subsection{Viktiga intressenter}
        \begin{itemize}
            \item \textbf{Helsingborg Arena och Scen}.
            \\
            \item \textbf{Utvecklarna av verktyget}. Vi som jobbar med projektet.
                \\
            \item \textbf{Campus Helsingborg}. Våra handledare
        \end{itemize}
        
    \subsection{Mindre viktiga intressenter}
    \begin{itemize}
            \item \textbf{Företag som planerar event i Helsingborg}.
                \\
            \item \textbf{Robin Nilsson}. Verksamhetsledare på Campus Vänner.
            
    \end{itemize}
    
    \subsection{Användare och utvärdering av nyckelintressenter}
    
    \begin{itemize}
        \item Vilka användare har vi i systemet?
            \begin{itemize}
                \item [--] \textbf{Slutanvändarna}, det vill säga supply chain. Användarna som kommer rappotera in data från sitt företag. 
                \item[--] \textbf{Malin Hollgren} kommer att vara admin av system samt extrahera statistiken som verktyget sammanställer. 
            \end{itemize}
        \item Utvärdering av nyckelintressenter
            \begin{itemize}
                \item [] \textbf{Malin Hollgren}, projektledare för möten och kongresser på Helsingborg Convention and Event Bureau
                    \begin{itemize}
                        \item[--] \textbf{Fackkunskap :} Kunskap om verktygets funktion och hur statistiken kommer att användas.
                        \\
                        \item[--] \textbf{Nuvarande arbetssätt :} Tillförser oss med nödvändig information kring produkten, slutanvändarna samt generell information.
                        \\
                        \item[--] \textbf{Behov :}Ett lätthanterligt verktyf som genererar statistik från användarna som sedan enkelt kan kopieras och skickas in till GDSM.
                        \\
                        \item[--] \textbf{Önksemål :}Kunna se enskilda inrapporteringar från företag.
                        \\
                        \item[--] \textbf{Prioriteringar :}Ett slutgiltigt verktyg som gör fungerar utifrån behovet, samt att det ska vara enklet för slutanvändarna.
                        \\
                    \end{itemize}
           
                \item [] \textbf{Global Destination Sustainability Movement}, slutstationen för den statistik vårt verktyg samlar in.
                    \begin{itemize}
                        \item[--] \textbf{Fackkunskap :} Hur frågor ska tolkas och hur det ska rapporteras in från företagen.
                        \\
                        \item[--] \textbf{Nuvarande arbetssätt :} Hjälpa oss att förstå vissa frågor ska tolkas.
                        \\
                        \item[--] \textbf{Behov och prioriteringar :} Statistik som passar deras format.
                        \\
                    \end{itemize}
                    
                     \item [] \textbf{Malin Planander, Miljöbron}
                    \begin{itemize}
                        \item[--] \textbf{Fackkunskap :}Hur arbetet med företaget ska gå framåt på bästa sätt. Generell kunskap om liknande projekt/sammarbeten.
                        \\
                        \item[--] \textbf{Nuvarande arbetssätt :} Hjälper oss med hålla kontakten med företaget, bidrar med generell kunskap om projektet samt bokar in framtida möten med oss och företaget.
                        \\
                        \item[--] \textbf{Behov, önskemål och prioriteringar :}Att projketet går så smidigt som möjligt och att det blir en lyckad slutprodukt.
                        
                    \end{itemize}
            \end{itemize}
              
        \subsection{Personas}
        
        \begin{itemize}
            \item [] \textbf{Joakim Gröte}
            \\
            Joakim Gröte är 45 år gammal och har ett nystartat familjeföretag med sin fru inom restaurangbranschen, detta har varit hans dröm hela livet. Han är  nyinflyttad i en lägenhet i Helsingborg med henne och sitt barn. Joakim har mindre erfarenhet med datorer eftersom han hela sitt liv valt att fokusera på köket och matlagning. Mer än att surfa på nätet och betala räkningar har han i stort sätt aldrig gjort. När det kommer till mer avancerade datorrelaterade uppgifter låter han sin fru ta hand om detta. Efter stressiga dagar i köket är det sista Joakim vill att sätta sig framför ett mer komplicerat datorprogram. 
            
             \item [] \textbf{Frida}
            \\
            Frida, 40. Jobbar som miljöansvarig på Elite Hotel i Helsingborg. Hon är tekniskt kunnig men har begränsat med tid på grund av att hon har många bollar i luften. Hon jobbar hemifrån och har två barn som hon måste vara tillgänglig inför och de behöver hennes uppmärksamhet. Samtidigt är hon mån om Elites rykte som ett hotell som försöker uppnå hållbarhet men hon ogillar enkäter starkt.
            
             \item [] \textbf{Kurt}
            \\
            Kurt, 55, gör pizza men ser dåligt. Han är egenföretagare. Eftersom många av hans bagare inte tar jobbet på allvar måste han vara tillgänglig överallt. Det sista han vill är att lägga en massa tid på svara på enkäter. Pga sina barnbarn är han insatt i Greta Thunbergs arbete. Han tänker därför svara på enkäten han har fått gällande hållbarhetsutveckling då han sitter på toa.
            
            \item [] \textbf{Jocke}
            \\
            Jocke, 35, jobbar på konsert- och eventlokalen Tivoli i Helsingborg som programansvarig. Inga barn. Han bryr sig inte så mycket om omvärlden utan har bara intresse för sig själv och missar lätt både viktiga och oviktiga mail. Det är hans chef som trycker på med hållbarhetsfrågor och vill att Jocke ska svara på en hållbarhetsenkät. Han tänker ta sig genom enkäten snabbt eftersom han har lite tålamod. Eftersom han ofta får leveranser finns en risk att han blir avbruten mitt i det han gör. Han jobbar inte så gärna från sitt skrivbord utan är mer tillgänglig på mobilen. Han vill att Helsingborg Stad ska gilla Tivoli så att de kan få mer bidrag till sin verksamhet. 
            
            \item [] \textbf{Pelle Plutt}
            \\
            Pelle Plutt är 45 år och jobbar som Miljöansvarig för Scandic Hotels i Helsingborgsområdet. Pelle bor i en 3a i centrala Helsingborg med sin sambo Karin Kanon. Pelle brukar gå eller köra elscooter till jobbet. Pelle har ett eget kontor i Scandics huvudkontor i Helsingborg där han har tillgång till en pc, samt en IT-avdelning för stöd vid IT-ärenden. Pelle försöker dock undvika att kalla på IT-avdelningen då han upplever dem som lite dryga. Han har relativt goda kunskaper i program som excel och word, däremot har han ingen djupare förståelse för hur datorer och program fungerar. Pelle gillar inte att sätta sig in i nya system då han ofta tycker det är krångligt.
            
             \item [] \textbf{Bogdan Mylleberga}
            \\
            Bogdan Mylleberga är 43 år och arbetar som Miljöansvarig på Ängelholm-Helsingborg Airport. Bogdan är mån om och ansvarar över flygplatsens alla miljöaspekter. Allt från val av fossilt bränsle, till återvinning av plastflaskor. Han jobbar mestadels från sin dator med att sammanställa och analysera data som samlas in från flygplatsen. Eftersom han arbetar mycket med datorn kan Bogdan arbeta hemma en hel del, vilket är lyckosamt eftersom Bogdan blev pappa för sex månader sedan. Nu kan han umgås mer med sin dotter Aada och sin mammalediga fru Kharin.

            Bogdan är bekymrad över att de datorsystem som han tvingas använda på arbetet börjar bli så pass utdaterade att de har svårt att hantera ny data i takt med teknologins framfart. Systemet är alltså anpassat efter gammal teknik. Bogdans högsta önskning är att ett enkelt och smidigt rapporteringssystem, som kunde samla in flygplatsens miljöhantering, är på ingång.


            

            


            
        \end{itemize}
        
        
              
       
    \end{itemize}
        
    
    
    \newpage
    
\section{Krav}
    \subsection{Funktionella krav}
    
    \subsection{Design krav}
    
    \subsection{Projekt krav}

\end{document}