\documentclass{article}
\usepackage[utf8]{inputenc}

\title{Övning-5-delgrupp-1}
\author{David Vilppu}
\date{September 2021}

\begin{document}

\maketitle

\section*{Uppgift 1}
\subsection*{Vilka personer/roller ska läsa kravspecifikationen?}
Christin, samtliga projektmedlemmar, Malin Planander samt Malin Hollgren.
\subsection*{Vilken kunskap har dessa om krav?}
Projektmedlemmarna har lekmannamässig kunskap då vi precis påbörjat en kurs inom kravhantering. Somliga har dock erhållit kunskap sedan PUSP-kursen, där en kravspecifikation utformades.\\\\
Malin Hollgren arbetar som projektledare och kan därför förväntas ha kunskap om kravspecifikationer samt kravhantering. Det är hon som i samhötighet med oss utformar kraven efter hennes önskemål och behov. \\\\
Christin är kursledare inom kravhanteringskursen och får därmed anses ha mycket god kunskap. \\\\
Då Malin Planander har varit med vid flera liknande projekt genom åren kan vi anta att hon har god kunskap om hur en kravspecifikation bör skrivas


\section*{Uppgift 2}
\subsection*{Hur ska kraven specificeras? Dvs hur ska kravspecifikationen byggas upp och se ut?}
Vi har för tillfället valt att börja kategorisera kraven i funktionella-, data och kvalitetskrav. Senare kommer dessa kraven att grupperas utifrån deras områden inom produkten. Till exempel kan datakrav och funktionella krav som berör användarvänlighet grupperas tillsammans.
\begin{enumerate}
    \item För datakraven kommer ER-diagram, virtual windows, kontextdiagram och data dictionary användas.
    \item För de funktionella kraven kommer scenarion användas.
    \item För kvalitetskraven kommer vi använda oss av "open metric and open target" från kurslitteraturen. Dessutom kommer vi använda oss av kapacitets-, pricksäkerhets-, användbarhets-, säkerhets-, support- och prestandakrav
\end{enumerate}


\section*{Uppgift 3}
\subsection*{Vilka metoder ska användas för att validera kraven i projektet?}

De metoder som kan användas för att validera de krav vi utformar är
\begin{enumerate}
    \item Tester (prototyptester och genomgångar)
    \item Granskningar
\end{enumerate}

\section*{Uppgift 4}
\subsection*{1. Hur ska kraven prioriteras?}
I programmet är det viktigt att datan sammanställs korrekt, vilket innebär att dessa kraven är viktigast. Ett av våran mål med arbetet är att så många som möjligt rapporterar in via verktyget, Vilket betyder att även mål om användarvänlighet prioriteras. 

\subsection*{Vilka krav ska prioriteras?}
Följande är exempel på krav som ska prioriteras
\begin{enumerate}
    \item 4 av 5 användare ska kunna slutföra rapporteringen inom 3 minuter(open ended metric).
    \item Användare loggar in till formuläret med sin engångskod (scenario).
    \item Datan skall sammanställas på korrekt sätt (enligt GDSMs krav) (data dictionary).
    \item Engångskoden förbrukas när användaren skickar in sin rapport (scenario).
\end{enumerate}
\subsection*{Vilken skala ska ni prioritera efter?}
A-krav, B-krav, C-krav.
\end{document}
