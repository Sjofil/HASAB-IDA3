\documentclass{article}
\usepackage[utf8]{inputenc}

\title{Övning-5-delgrupp-1}
\author{David Vilppu}
\date{September 2021}

\begin{document}

\maketitle

\section*{Uppgift 1}
\subsection*{Vilka personer/roller ska läsa kravspecifikationen?}
Christin, samtliga projektmedlemmar, Malin Planander samt Malin Hollgren.
\subsection*{Vilken kunskap har dessa om krav?}
Projektmedlemmarna har lekmannamässig kunskap då vi precis påbörjat en kurs inom kravhantering. Somliga har dock erhållit kunskap sedan PUSP-kursen, där en kravspecifikation utformades.\\\\
Malin Hollgren har kanske inte några djupa kunskaper om kravspecifikationer och kravhantering. Men det är hon som i samhörighet med oss utformar kraven efter hennes önskemål och behov, vilket hon har god kunskap om. \\\\
Christin är kursledare inom kravhanteringskursen och får därmed anses ha mycket god kunskap. \\\\
Malin Plananders kunskap om kravhantering och krav är känner vi inte till. Däremot har hon varit med om ett flertal projekt där kravhantering troligtvis varit inkluderat.

\section*{Uppgift 2}
\subsection*{Hur ska kraven specificeras? Dvs hur ska kravspecifikationen byggas upp och se ut?}
Vi har för tillfället valt att kategorisera kraven i funktionella-, data och kvalitetskrav.
\begin{enumerate}
    \item För datakraven kommer ER-diagram, virtual windows och data dictionary.
    \item För de funktionella kraven kommer scenarion användas
    \item För kvalitetskraven kommer troligtvis endast vanlig kravtext användas.
\end{enumerate}


\section*{Uppgift 3}
\subsection*{Vilka metoder ska användas för att validera kraven i projektet?}

De metoder som kan användas för att validera de krav vi utformar är
\begin{enumerate}
    \item Tester
    \item Granskningar
\end{enumerate}

\section*{Uppgift 4}
\subsection*{1. Hur ska kraven prioriteras?}
De krav som kategoriseras som funktionella prioriteras högst, det innefattar bl.a. datakrav eftersom det är viktigt för funktionen att in- och utdata är korrekt.\\\\
Utöver de funktionella kraven krävs att användarvänligheten når godkänd nivå, vilket får även ses som en prioritering för att kunna leverera ett godkänt system.

\subsection*{Vilka krav ska prioriteras?}
Följande är exempel på krav som ska prioriteras
\begin{enumerate}
    \item 4 av 5 användare ska kunna slutföra rapporteringen inom 3 minuter.
    \item En användare ska kunna ange sin engångskod för att kunna starta rapporten på startsidan.
    \item Den data som anges av användaren ska beräknas utifrån GDSM's formulär.
    \item 
\end{enumerate}
\subsection*{Vilken skala ska ni prioritera efter?}
1-10.
\end{document}
