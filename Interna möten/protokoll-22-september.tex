\documentclass{article}
\usepackage[utf8]{inputenc}

\title{mötesprotokoll 22 september}
\author{anna.v.bergvall }
\date{September 2021}

\begin{document}

\maketitle


Närvarande: Oscar, Filip, Pontus, Sabah, Anna och David.


\begin{itemize}
    \item Filip går igenom sin mockupdesign. Bra poäng att man får välja om man vill se mer info.
    \item David visar sin design. Viktigt med "progress bar". Ska vara enkelt att trycka med tummen. En bekräftelse och tack ska skickas ut när enkäten har fyllts i och skickats iväg. 
    \item Sabah visar. Mer som en storyboard och har även med Malins adminsida. 
    \item Oscar visar. Engångskod, stora knappar och ingen onödig text är bra.
    \item Anna visar. Det kan finnas en poäng att ställa enklare frågor till användarna och att procentsatserna räknas ut av vårt program. 
    \item Pontus visar. Mer fokus på admin. Automatisk mail med engångslänk kan användas. Maillista där varje adress är knuten till en branch.
    \item Frågor till Malin Hollgren formuleras.
    \item Databas och webbservrar: David lägger upp ett dokument på GitHub där vi kan fylla i intressanta webservrar.
    \item Vi gör tre nya textkanaler i Discord: datakrav, funktionella-krav och kvalitetskrav. Här går det bra att brainstorma.
    \item Vi delar upp arbetet. David, Oscar och Filip får datakraven och Anna, Pontus och Sabah får i uppgift att formulera funktionella krav. 
    \item Alla ska skicka in kravprocessboken. Bestäms genom lottning från och med nu.
\end{itemize}
    \\
    
    Vid pennan: Anna

\end{document}
