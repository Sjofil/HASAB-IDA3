\documentclass[12pt]{article}
\usepackage[utf8]{inputenc}
\usepackage{fancyhdr}
\usepackage{graphicx}
\usepackage{geometry}
\usepackage{float}
\usepackage[utf8]{inputenc}
 
\usepackage{array}
\usepackage{makecell}

% ---- Commands ------- %
\newcommand{\documentNumber}[1]{
    \LARGE  \textbf{Slutrapport}
    \\
    \medskip
}
\newcommand{\documentVersion}[1]{
    \medskip
}
\newcommand{\documentTitle}[1]{
    \centerline{\rule{13cm}{0.4pt}}
    \bigskip \bigskip
    \LARGE \textbf{MAMF40} \\
    \bigskip
    \LARGE {#1} \\
    \bigskip \bigskip
    \centerline{\rule{13cm}{0.4pt}}
}

\newcommand{\documentDate}[1]{
    \date {#1} 
}

\renewcommand\theadalign{bc}
\renewcommand\theadfont{\bfseries}
\renewcommand\theadgape{\Gape[5pt]}
\renewcommand\cellgape{\Gape[5pt]}


\renewcommand{\contentsname}{Innehållsförtäckning}

% --- Header & Footer ---- %
\pagestyle{fancy}
\lhead{\leftmark}
\rhead{}
\rfoot{\thepage}
\cfoot{}
\lfoot{}


% ------------------------------------------------ #

% ----- FILL THIS ----- %
\title {
    \documentNumber {01}    

    % Full name - SHORTNAME
    \documentTitle {Helsingborg Event and Convention Bureau}
    
    % Format: YYYY-MM-DD
    \documentDate {2021-09-29}
    \documentVersion Vv 0.1
    
    \author{Anna Bergvall - Oscar Blixt - Pontus Persson - Filip Sjövall - David Vilppu - Sabah Zafar }
}

\begin{document}
\addtocontents{toc}{\protect\setcounter{tocdepth}{2}}
\maketitle

\thispagestyle{empty}



\newpage

\tableofcontents


\newpage

\section{Dokument Historia}
\begin{tabular}{ l | l | l }
    Version & Datum & Beskrivning \\
    \hline
    0.1 & 2021-09-29 & Dokumentet skapat. \\
     \hline
   
   
\end{tabular}

\newpage
% ----- SKRIV UNDER VARJE TITEL ----- %

\section{Division of work}
On the inside of the cover page, you should make a named description of how the division of labor in the project has been – e.g., who has worked with different parts of the prototype(s) and the report. Keep a reasonable level of detail! This is a good preparation for your thesis project where you need to do this as well.

\section{Abstract}
A maximum of 250 words!Brief summary. It should be enough to read abstracts to get a reasonable understanding of what has been done and how.

\section{Table of Contents}
Table of contents, down to heading level 2.

\section{Introduction}
Short background, what is the basis for this project, for example an account of the project definition from the company and other relevant background.

Vår kund Helsingborg Convention and Event Bureau(HCEB) önskade att gå med i det globala nätverket Global Destination Sustainability Movement (GDSM). Som medlem i nätverket rapporteras det årligen in hållbarhetsdata som bidrar till en positiv utveckling av staden som en mötes- och evenemangsdestination. Målet med projektet är att utveckla ett digitalt rapporteringsverktyg som samlar in och sammanställer hållbarhetsdata från aktörer inom besöksnäring i städer och kommuner. Rapporteringsverktyget erbjuder en skräddarsydd sammanställning och större kontroll över distribution och administration av den insamlade datan. Det utvecklade rapporteringsverktyget är en webbaserad hemsida som ska användas av aktörer med olika tekniska färdigheter, vilket gör att stora krav ställs på användarvänlighet.





\section{Process}
Needs to be divided in the sections: Planning, development, and risk analysis.

\subsection{Planning}
Relevant accounts of the planning process, including project model with motivations for choosing that specific model, time planning and other relevant information.\\\\
En fördel med ett agilt arbetssätt där det finns en tät kontakt med uppdragsgivare är att det upprätthåller en regelbunden kommunikation mellan kunden och utvecklingsteamet där fungerande programvara prioriteras över utförlig dokumentation. Dessutom värderas möjlighet för anpassning och förändring. Projektgruppen har därför valt att tillämpa Scrum. [1] \\\\
Projektets mål var att utveckla ett fungerande rapporteringsverktyg som ska användas av Helsingborg Convention and Event Bureau (HCEB) för sammanställning av \\ hållbarhetsdata från aktörer verksamma inom besöksnäringen i Helsingborg. \\\\
Utifrån målbildanalysen valdes en lättrörlig projektmodell eftersom informationen och kunskapen om projektet i det inledande skedet var knapp. Vi ville snabbt kunna agera och ta beslut om arbetet skulle gå i fel riktning vilket det kan finnas risk för i början. Projektgruppen ville på det sättet ha frihet att kunna ta beslut i efterhand om det exempelvis skulle dyka upp nya krav. Förhoppningen var att detta skulle leda till en mer flexibel och anpassningsbar utveckling. Av denna anledning beslutades det att inledningsvis sätta våra sprints till en vecka. \\\\ 
En annan anledning till att arbeta med en lättrörlig modell var möjligheten att hela tiden jobba med en testbar produkt istället för att bara arbeta med en slutprodukt. Med andra ord var målet att jobba med "evolutionary prototypes" (Interaction Design-boken) som inte skulle kasseras mellan varje användartest utan snarare vidareutvecklas enligt kundens önskemål.\\\\
Senare i projektet jobbade vi mer med en iterativ process eftersom vi hade då fått en tydligare målbild och kunde strukturera upp arbetet. Vi valde då att jobba utifrån en scrum board där den uppgift som fanns högst upp på scrum boarden skulle bli tagen först. \\\\
För att säkerställa att alla gruppmedlemmar var uppdaterade med rätt information hade vi kommunikation på discord. Vi hade även många inplanerade möten internt inom projektgruppen för att kunna ta viktiga besult och diskutera utvecklingsarbetet. Kunden var också involverad i vårt arbete med en kontinuerligt kontakt vid förutbestämda möten med jämna mellanrum för att få en återkoppling.  


\subsection{Development}
A structured description of the process and important decisions. Use test results or theory to argue for your decisions. Divide section in some practical manner, for example in sprints. Name them in a good way (not sprint 1 2 3...). Don’t try to make a description of every single decision or feature but make an informed selection of important and interesting things to present.


Det centrala fokuset i Scrum läggs på att skapa små delmål under tidsbegränsade perioder de så kallade sprintar. Delmålen i sprintarna ska fullgöra uppsättning av slut målen i projektet. Man får på det sättet en klar fördelning av arbetsuppgifter som teammedlemmar kan utföra och ansvara för. Vi anpassade att rollen av scrum master skulle delas mellan två projektmedlemmar. Det beslutet togs för att inte överbelasta ansvaret på enskilt person. 

En fördel med scrum är det retrospektiva att teammedlemmar kan reflektera runt projektprocessen. Man kan ta upp det som fungerat bra och det som behöver förbättras i processen. Detta tillåter projektmedlemmarna att föra en diskussion och ger utrymme till den enskildes åsikter kring projektet. Det gör att mer anpassade och realistiska framtida sprintar kan skapas utifrån förslaget från den gemensamma öppna diskussionen. 

En annan fördel med scrum är att det främjar och öppnar upp för kommunikation mellan kunden och utvecklingsteamet för att enklare anpassa arbetet utifrån kundens behov som kan komma att förändras. På så sätt kan mycket tid komma att sparas eftersom nya krav kan tilläggas eller tas bort utan att spendera allt för mycket tid på dokumentationer. Man vill med detta att fokuset istället ska hela tiden ligga på utvecklingsarbetet. 

\subsection{Risk analysis}
Presentation of original risk analysis with motivations, your plan for updating said risk analysis, and an account on the revisions made during the process. \\\\
 Tidigt i projektet, med hjälp av artikeln \textit{Riskanalys} [4], implementerades miniriskmetoden för att kunna identifiera riskerna och därefter analysera dem. En uppskattning av sannolikheten att dessa risker skulle kunna inträffa och vilka konsekvenserna följd av riskerna gjordes.\\\\
 Syftet med riskanalysen var att ta fram en åtgärdsplan som skulle följas vid hantering av risker, både proaktivt men även då eventuella risker är ett faktum. I riskanalysen skapades en lista över \textit{projektrisker, produktrisker} och \textit{affärsrisker}.\\\\
 Den största projektrisken som identifierades var en bristfällig kommunikation bland projektmedlemmarna. För att minimera denna risk skapades en tydlig kommunikationsplan samt grundregler. Denna åtgärd skulle leda till en bättre stämning i gruppen genom att missförstånd minimeras.\\\\
 Den största produktrisken som identifierades var att produkten inte skulle vara tillräckligt intuitiv, dvs. att användbarheten var för låg. För att minimera denna risk beslutades att utföra många användartester. Gruppen ansåg att detta skulle ge en god insikt om hur användare upplever systemet.\\\\
 Projektgruppen beslutade att i början av varje sprint koppla riskerna till de delmoment som sprinten innehöll för att kunna identifiera eventuella risker i den specifika sprinten. Om en gruppmedlem stötte på risker skulle detta meddelas under veckovisa möten, där riskerna kunde ställas mot åtgärdsplanen för att kunna hantera dessa omgående.\\\\
 





Skriv om revisions





\section{Result}
Presentation of final prototype. Limited number (good selection) of images / screen dumps. Focus on interaction and functionalities, but also technical issues, without going too deep into details. In short, describe the projects technical set-up, where you borrowed code, frameworks etc. As far as possible, use the same way of referencing as for text. That is, try to identify an author (if there is only a pseudonym or user, it is better than nothing), a title and then link to website or repository.

\section{Discussion}
Sections on reflections about the process, methods and result of the project work. Subheadings are encouraged, these need not be checked with supervisor. And finally: to what degree does the product solve the original problem? Motivate!

\section{Advice for students next year}
Present the 5 most important advice you would like to give students who study this course next year (discuss in group and rank them prior to documentation).Conclusion

\section{Conclusion}
Brief conclusion: What did you accomplish? What made you succeed or not? 

\section{References}





\bibliographystyle{alpha}


\end{document}