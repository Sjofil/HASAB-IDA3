\documentclass[12pt]{article}
\usepackage[utf8]{inputenc}
\usepackage{fancyhdr}
\usepackage{graphicx}
\usepackage{geometry}
\usepackage{float}
\usepackage[utf8]{inputenc}
\usepackage{authblk}
 
\usepackage{array}
\usepackage{makecell}

% ---- Commands ------- %
\newcommand{\documentNumber}[1]{
    \LARGE  \textbf{Slutrapport}
    \\
    \medskip
}
\newcommand{\documentVersion}[1]{
    \medskip
}
\newcommand{\documentTitle}[1]{
    \centerline{\rule{13cm}{0.4pt}}
    \bigskip \bigskip
    \LARGE \textbf{MAMF40} \\
    \bigskip
    \LARGE {#1} \\
    \bigskip \bigskip
    \centerline{\rule{13cm}{0.4pt}}
}

\newcommand{\documentDate}[1]{
    \date {#1} 
}

\renewcommand\theadalign{bc}
\renewcommand\theadfont{\bfseries}
\renewcommand\theadgape{\Gape[5pt]}
\renewcommand\cellgape{\Gape[5pt]}


\renewcommand{\contentsname}{Innehållsförtäckning}

% --- Header & Footer ---- %
\pagestyle{fancy}
\lhead{\leftmark}
\rhead{}
\rfoot{\thepage}
\cfoot{}
\lfoot{}


% ------------------------------------------------ #

% ----- FILL THIS ----- %
\title {
    \documentNumber {01}    

    % Full name - SHORTNAME
    \documentTitle {Helsingborg Event and Convention Bureau}
    
    % Format: YYYY-MM-DD
    \documentDate {\today}
    \documentVersion Vv 0.1
    
\author{Anna Bergvall - Oscar Blixt - Pontus Persson\\\\ David Vilppu - Filip Sjövall - Sabah Zafar}



}

%\author{
%  Anna Bergvall\\
%  \and
 % Filip Sjövall\\
%   \and
 % David Vilppu\\
 %   \and
 % Oscar Blixt\\
%    \and

% Sabah Zafar\\
 %   \and
 % Pontus Persson\\
%}






\begin{document}
\addtocontents{toc}{\protect\setcounter{tocdepth}{2}}
\maketitle
%%\end{document}
\thispagestyle{empty}



\newpage

\tableofcontents



\newpage

%\ dokument historia ska inte vara med
%section{Dokument Historia}
%\begin{tabular}{ l | l | l }
 %   Version & Datum & Beskrivning \\
  %  \hline
 %   0.1 & 2021-09-29 & Dokumentet skapat. \\
 %    \hline
   
   
%\end{tabular}


\newpage
% ----- SKRIV UNDER VARJE TITEL ----- %

\section{Arbetsfördelning}
On the inside of the cover page, you should make a named description of how the division of labor in the project has been – e.g., who has worked with different parts of the prototype(s) and the report. Keep a reasonable level of detail! This is a good preparation for your thesis project where you need to do this as well.
    

\textbf{Anna}: Sammanställt kriterierna som de facto ska vara med i rapporteringsverktyg i en wordfil. Implementerat metoderna login och loginAdmin i flask. Planerat grafisk schema för arbetsveckor. Haft ansvar för kommunikationen med Malin Planander och Malin Hollgren. Skrivit på slutrapporten(planering). Skissat på designen för prototypen. Planerat sprintar och skapat en PowerPoint for delredovisning.

\textbf{Filip}: Planerat sprintar, implementerat för samtliga sidor Html och CSS kod template för frontend, Funktionalitet för kunna gå mellan sidor. Implementerat logiken för knapp-inputs. Skrivit på slutrapporten(utveckling). Skissat på designen för prototypen. 

\textbf{David}: Implementerat funktioner via JS/HTML, Tagit fram E/R diagram, Försökt göra designen mer konsekvent. Skrivit på slutrapporten(Riskanalys), skissat på designen för prototypen. 

\textbf{Oscar}: Implementerat databasen. Lagt in frågor i databasen. Implementerat funktionalitet för användarlogin, admin login och logout. Tagit fram E/R diagram. Skissat på designen för prototypen.

\textbf{Sabah}: Processrapport, arbetsrapport(1 och 2). Skrivit på slutrapporten(introduktion, process, planering, utveckling, riskanalys, resultat, diskussion). Implementerat startsidan med Html och CSS. Skissat på designen för prototypen.  

\textbf{Pontus}: Gjort prototyp för admin att maila ut länkar, lagt till blueprints i Flask för enklare samarbete. Flask sessions. Skissat på designen för prototypen.

\newpage
\section{Abstract}
A maximum of 250 words!Brief summary. It should be enough to read abstracts to get a reasonable understanding of what has been done and how.
\\\\
Sustainability is becoming an increasingly important part of businesses’ and organisation’s values. As a result, a wide array of sustainability certificates are attainable for showing environmental good-will and to support continuous efforts. One of these certificates is Global Destination Sustainability Movement (GDSM). The city of Helsingborg, a national hub for events and conferences, is aiming to acquire this certificate. In order to be a part of GDSM, cities must fill in a form containing 69 criteria ”that measures, benchmarks and improves the sustainability strategy and performance of tourism and events destinations” (GDS-Index 2021, Methodology).\\\\
For the purpose of collecting sustainability data from hotels and restaurants in Helsingborgs stad, a web based questionnaire was developed. The data was saved in a database using MySQL through the Flask API, written in Python. To ensure a mobile friendly and responsive application, the system was built using the CSS framework Bootstrap. 
Given the fact that the success of the project was dependent on response frequency, making the system easy to use for users of various technichal backgrounds was paramount. 

(and then something about user testing maybe)

\newpage
\section{Introduktion}
Short background, what is the basis for this project, for example an account of the project definition from the company and other relevant background.\\\\
Vår kund Helsingborg Convention and Event Bureau(HCEB)\footnote{https://helsingborgceb.com/} önskade att gå med i det globala nätverket Global Destination Sustainability Movement (GDSM). Som medlem i nätverket rapporteras det årligen in hållbarhetsdata som bidrar till en positiv utveckling av staden som en mötes- och evenemangsdestination. Målet med projektet är att utveckla ett digitalt rapporteringsverktyg som samlar in och sammanställer hållbarhetsdata från aktörer inom besöksnäring i städer och kommuner. Rapporteringsverktyget erbjuder en skräddarsydd sammanställning och större kontroll över distribution och administration av den insamlade datan. Det utvecklade rapporteringsverktyget är en webbaserad hemsida som ska användas av aktörer med olika tekniska färdigheter, vilket gör att stora krav ställs på användarvänlighet.







\section{Process}
Needs to be divided in the sections: Planning, development, and risk analysis.

\subsection{Planering}
Relevant accounts of the planning process, including project model with motivations for choosing that specific model, time planning and other relevant information.\\\\
En fördel med ett agilt arbetssätt där det finns en tät kontakt med uppdragsgivare är att det upprätthåller en regelbunden kommunikation mellan kunden och utvecklingsteamet där fungerande programvara prioriteras över utförlig dokumentation. Dessutom värderas möjlighet för anpassning och förändring. Projektgruppen har därför valt att tillämpa Scrum. [1] \\\\
Projektets mål var att utveckla ett fungerande rapporteringsverktyg som ska användas av Helsingborg Convention and Event Bureau (HCEB) för sammanställning av \\ hållbarhetsdata från aktörer verksamma inom besöksnäringen i Helsingborg. \\\\
Utifrån målbildanalysen valdes en lättrörlig projektmodell eftersom informationen och kunskapen om projektet i det inledande skedet var knapp. Vi ville snabbt kunna agera och ta beslut om arbetet skulle gå i fel riktning vilket det kan finnas risk för i början. Projektgruppen ville på det sättet ha frihet att kunna ta beslut i efterhand om det exempelvis skulle dyka upp nya krav. Förhoppningen var att detta skulle leda till en mer flexibel och anpassningsbar utveckling. Av denna anledning beslutades det att inledningsvis sätta våra sprints till en vecka. \\\\ 
En annan anledning till att arbeta med en lättrörlig modell var möjligheten att hela tiden jobba med en testbar produkt istället för att bara arbeta med en slutprodukt. Med andra ord var målet att jobba med "evolutionary prototypes" (Interaction Design-boken) som inte skulle kasseras mellan varje användartest utan snarare vidareutvecklas enligt kundens önskemål.\\\\
Senare i projektet jobbade vi mer med en iterativ process eftersom vi hade då fått en tydligare målbild och kunde strukturera upp arbetet. Vi valde då att jobba utifrån en scrum board där den uppgift som fanns högst upp på scrum boarden skulle bli tagen först. \\\\
För att säkerställa att alla gruppmedlemmar var uppdaterade med rätt information hade vi kommunikation på discord. Vi hade även många inplanerade möten internt inom projektgruppen för att kunna ta viktiga besult och diskutera utvecklingsarbetet. Kunden var också involverad i vårt arbete med en kontinuerligt kontakt vid förutbestämda möten med jämna mellanrum för att få en återkoppling. Dessutom skapades en för projektgruppen gemensam mailadress för mailkommunikation med kunden under utvecklingen.


\subsection{Utveckling}
A structured description of the process and important decisions. Use test results or theory to argue for your decisions. Divide section in some practical manner, for example in sprints. Name them in a good way (not sprint 1 2 3...). Don’t try to make a description of every single decision or feature but make an informed selection of important and interesting things to present.\\\\
I projketets början togs beslutet att använda en scrumboarden. Inledningsvis sattes sprintarnas längd till en vecka med motivation att det skulle tvinga fram snabb feedback och ge en mer responsiv utveckling. Dock visade det sig att en vecka inte var ultimat som sprintperiod. Enligt Kniberg \& Skarin [2] brukar en sprint vara mellan 1-4 veckor. Därför togs beslutet att istället låta en sprint sträckta sig över två veckor. Tanken bakom sprintens längd var att det troligtvis skulle ta mer än en vecka att implementera samt att om tiden för sprintarna var för kort skulle för mycket tid behöva läggas på planering istället för arbete. \\\\
\textbf{Sprint ett, vecka 40}
Arbetet under första sprinten kretsade kring dokumentation och det var bestämt att ER-digaram samt flödesdiagram skulle framställas. Tidigt under sprintsen togs ett besult om att slopa UML-diagram där insperation från Matz, K [3] där det läggs fram att det vid agil utveckling behövs mindre dokument. 
I gruppen bestämdes också att sprintens längd skulle minskas från två till en vecka, detta för att för att samtliga medlemmar var osäkra på vad som faktiskt skulle göras och mer struktur behövdes för att komma framåt i projektet. \\\\
\textbf{Sprint två, vecka 43}\\\\
Efter \textbf{sprint ett} las arbetete med projektet på is på grund av tentaperioder, detta innebar också att inga beslut togs. Samtidigt som gruppen inte arbetade med projektet så fortsatte en mailkonversation mellan kunden och Malin Planander på miljöbron, här bollades ideér om nya frågor till formuläret initierade en liten orolighet i gruppen eftersom det var luddigt om dessa faktiskt skulle behöva vara med i den slutgiltiga produkten eller inte.\\\\
\textbf{Sprint tre, vecka 44}\\\\
Arbetet enligt sprints återupptogs igen och på grund av att tidsplaneringen skapades utan tanke på att ett uppehåll skulle ske under tentaperioden så hamnade gruppen efter i tidsplaneringen. Detta ledde till att vi mer och mer slutade arbeta utifrån scrumboarden och arbetade med uppgifter vi kommit överens om under mötet vi har i början av varje sprint/vecka. Det blev också en naturlig övergång med tanke på att den tid som egentligen var utsatt för planering istället gick åt till att koda. \\\\
\textbf{Sprint tre, vecka 45} \\\\
Det hade tidigare diskuterats hur användaren skulle verifiera sig för att endast ett svar från unik användare skulle registreras för sammanställning av hållbarhetsdata. Enligt agilt utveckling finns det utrymme för förändringar av nya krav och därför togs beslutet att nya krav skulle införas att användaren identifierar sig med ett mailadress. Det övervägdes noga att det skulle bli enklare att utgå från dessa nya kraven eftersom kunden hade en önskan om att det skulle vara enklare för de att inte behöva administrera mycket jobb för identifiering av användare och därför övervägde vi att inte utesluta kravet. XXX(ange källa)
Vi beslutade också att användaren skulle endast ha möjlighet att välja bland flervalsfrågor för enklare sammanställning av data. Detta stödjas och av [3] som menar att man endast ska visa nödvändig information för användarna.\\\\
\textbf{Sprint tre, vecka 46} \\\
Utvecklingsarbetet tog fart och funktionalitet för på adminsidan, admin login och logout,  vidareutvecklades. Knappen för att kunna navigera mellan olika sidor har också förbättrats. Funktionerna i JavaScript följdes upp och vi lagt in diverse frågetyper som ska vara möjligt för användaren att kunna svara på inför datainsamlingen.\\\\
I databasen kunde vi lägga information som berörde hotell, byråer, institutioner och flygplatser. Vi undersökning har gjort på hur vi kan lagra användarens fråga svar för att kunna använda detta för sammanställningen. Vi beslutade att använda cookies för att lagra användarens svar på besvarade frågor för att senare kunna sammanställa.\\\\
Ett viktigt beslutat under denna sprinten var att flytta vår scrum board från Atlassian till Trello, som ett digital verktyg till vår hjälp i kommande sprintar. Atlassian var krånglig att använda, den hängde sig mycket och det var mer krävande att göra ändringar på Atlassian. Vi tog därför beslutet att göra övergången då meningen är att vi flitigt ska ha nytta av det i vår utveckling utan hinder.\\\\
\textbf{Sprint tre, vecka 47} \\\
Under denna sprinten började arbetet för att kunna mobilanpassa webbplatsen samtidigt togs det fram en lösning för felidentifiering av användare. Databasen konstruerades om för att det skulle vara möjligt att hämta svarsalternativ från databasen som användaren kan sedan välja bland från webbplatsen. Samtidigt implementerades funktionaliteten för att kunna skicka in användarens svar till databasen. Svarsalternativen omstrukturerades så att användaren skulle endast använda sig av flervalsfrågor. Vi tog fram test scenario för att testa vår webbplats utifrån användbarhetsperspektiv och undersökning om enkäter och dess layout gjordes för att koppla teorin till verkligheten.\\\\
\textbf{Sprint tre, vecka 48} \\\
Under denna sprinten omstrukturerades databasen så att det endast skulle finnas en användare kopplad till en unik mailadress för identifiering. Funktionalitet för adminsidan vidare implementerades exempelvis kunde admin ta bort en användare från databasen och kunde även få en varning om samma användare lades till två gånger. En annan viktig ändring som gjordes i implementeringen var att det inte skulle vara möjligt att kunna navigera till en viss sida av webbplatsen genom ändring av URL:en. 


\subsection{Riskanalys}
Presentation of original risk analysis with motivations, your plan for updating said risk analysis, and an account on the revisions made during the process. \\\\
 Tidigt i projektet, med hjälp av artikeln \textit{Riskanalys} [4], implementerades miniriskmetoden för att kunna identifiera riskerna och därefter analysera dem. En uppskattning av sannolikheten att dessa risker skulle kunna inträffa och vilka konsekvenserna följd av riskerna gjordes.\\\\
 Syftet med riskanalysen var att ta fram en åtgärdsplan som skulle följas vid hantering av risker, både proaktivt men även då eventuella risker är ett faktum. I riskanalysen skapades en lista över \textit{projektrisker, produktrisker} och \textit{affärsrisker}.\\\\
 Den största projektrisken som identifierades var en bristfällig kommunikation bland projektmedlemmarna. För att minimera denna risk skapades en tydlig kommunikationsplan samt grundregler. Denna åtgärd skulle leda till en bättre stämning i gruppen genom att missförstånd minimeras.\\\\
 Den största produktrisken som identifierades var att produkten inte skulle vara tillräckligt intuitiv, dvs. att användbarheten var för låg. För att minimera denna risk beslutades att utföra många användartester. Gruppen ansåg att detta skulle ge en god insikt om hur användare upplever systemet.\\\\
 Projektgruppen beslutade att i början av varje sprint koppla riskerna till de delmoment som sprinten innehöll för att kunna identifiera eventuella risker i den specifika sprinten. Om en gruppmedlem stötte på risker skulle detta meddelas under veckovisa möten, där riskerna kunde ställas mot åtgärdsplanen för att kunna hantera dessa omgående.\\\\
 



\section{Resultat}
Presentation of final prototype. Limited number (good selection) of images / screen dumps. Focus on interaction and functionalities, but also technical issues, without going too deep into details. In short, describe the projects technical set-up, where you borrowed code, frameworks etc. As far as possible, use the same way of referencing as for text. That is, try to identify an author (if there is only a pseudonym or user, it is better than nothing), a title and then link to website or repository.\\\\
Vi utvecklade en enkelt webbapplikation grundad på Bootstrap eftersom då kunde vi bygga en responsiv design för webbsidan. Det ger användaren möjligheten att kunna få utkomst till webbsidan även från andra enheter än en dator såsom en mobil. Vi valde att jobba med Bootstrap eftersom det finns fördefinierade CSS-klasser, många komponenter för HTML-element samt stöd för JavaScript-plugins, vilket gjorde utvecklingen flexibelt för oss. Databasen mySQL lades med servern XXX för att vi skulle ha friheten att hantera både servern och databasen genom XXX oberoende av extern mjukvaruprogram. Databasens uppbyggnad kan återfinnas i figuren XXX i appendix XXX.\\\\
Utifrån kundens önskan hålls uppbyggnaden av webbapplikationen enkelt eftersom användaren förväntas ha varierande teknisk kunskap. Förutom att användaren kan med en förutbestämt unik inloggningskod rapportera data i form av svar på frågor som är aktuella för just deras bransch finns också en admin sida, se Appendix XXX. På admin sidan kan man sammanställa data och få en överblick av statistiken men även söka efter svar från en viss aktör se Appendix XXX. 

\section{Diskussion}
Sections on reflections about the process, methods and result of the project work. Subheadings are encouraged, these need not be checked with supervisor. And finally: to what degree does the product solve the original problem? Motivate!\\\\
Under projektets gång stötte vi på problem som vi inte hade räknat med från början. Det var krångligt att sätta upp databasen med mysql vid användning av Flask. Detta med anledning av att det fanns bristfällig dokumentationen på hur man kunde gå tillväga och tog därför lång tid att åtgärda felet. Ett annat problem som vi stötte på var att kunna skicka POST requests för servern vid implementering av inloggningsfunktionen. Vi hade också problem med att synkronisera vissa funktioner för hemsidan med Flask. Misstaget som gjordes här var att vi inte hade tänkt på att de implementerade funktionerna skulle även synkroniseras med Flask.\\\\
Vår tidsuppskattning som gjordes i början för utvecklingsarbetet för olika deluppgifter var inte noggran planerad. Vi hade inte ett tydligt tidsmål för hur mycket tid en teammedlem skulle lägga per sprint och därför kunde ibland vissa uppgifter antingen bli klara mycket fortare än räknat med eller ta ännu längre tid. Genom att blicka tillbaka tänker vi oss att det har varit en lärande process som lärt oss vikten av noggrann planering och att försöka göra mer realistiska tidsuppskattningar. Vi hade inte heller räknad med att under nästan två veckors period var både tentaplugg och tentaperioden. Det ledde till att vi hamnade lite efter i planeringen och fick därför lägga mer tid för att ligga i fas igen för att undvika förseningar.\\\\
Backloggen utnyttjades mestadels för tekniska utvecklingen och har främjat arbetsflödet. Det gav en överblick över utförd jobb och så att vi försökte att hålla deadlines.\\\\
Hade vi brist på kompetens inom något område????? diskutera det här ........... \\\\
%- Test av HiFo prototyp - flytta ett en annan sektion men diskutera under Diskussion 
\textbf{Test av HiFo prototyp }\\\
För att utvärdera den slutliga prototypen skapades några testscenarier med frågor som rör användbarhet och användarupplevelse. Observationstest planerades på ett sätt där en  testperson skulle interagera med prototypen av webbplatsen baserat på testscenarierna utan att förklara sitt agerande och utan ett ingripande av observeraren. Det för att låta testpersonen göra val baserat på deras naturliga användning av hur de i verkligheten skulle svara på frågeformuläret på vår utvecklade rapporteringsverktyg. Dessa interaktioner skulle sedan dokumenteras och  analyseras.\\\\
Under en testsession gav en av projektmedlemmarna uppgifter till en person tänkt som en användare och resultatet observerades. Detta gav en insikt att kunna jämföra testpersonens agerande med gruppens förväntningar för att vid behov kunna korrelera det som inte varit så intuitivt.\\\\
Användbarhetstesterna gjordes av totalt 4 försökspersoner vars bakgrund var variende både gällande ålder och datorkunskaper. En del av försökspersonerna jobbade dagligen med datorer medan andra hade måttlig interaktion med en dator och var flitiga användare av mobiltelefonen. \\\\
Försökspersoner valdes med baktanken om att de skulle vara av varierande karaktär för att mer rättvist spegla vår slutanvändare. Efter en testsession ställdes öppna frågor som till exempelvis "Var det något du tänkte på under tiden du svarade på frågorna?", varpå deras svar noterades och analyserades tillsammans med samtliga gruppmedlemmar. Användbarhetstestningen var grunden för att kunna ändra innehåll och göra det konsekvent samt uppenbar för användaren vad det gäller exempelvis namnval av knappar eller hur informationen var tillgänglig för användaren.

\section{Råd till studenter för kommande år}
Present the 5 most important advice you would like to give students who study this course next year (discuss in group and rank them prior to documentation).Conclusion\\\\
Våra fem råd till studenter som ska läsa kursen kommande året hade varit nedanstående:
\begin{enumerate}

\item  Kommunikationen mellan projektmedlemmarna är viktigt och underlättar arbetet enormt. Om något känns oklart var inte rädda för att ställa frågor för en bättre förståelse. Ha som utgångspunkt att syftet med projektprocessen är att lära sig för att längs vägen nå ett slutmål där alla inom gruppen vill lyckas.
\item Prata tidigt om varandras kompetenser och styrkor för att utnyttja dessa i projektet på bästa sätt. Ta hjälp av varandra och våga utmana er själva för att utvecklas inom något. 
\item  Planera tidigt att jobba tillsammans och inför schemalagda arbetspass där alla kan vara med för att jobba parallellt på olika deluppgifter. På det sättet kan alla bidra med något och arbetet rullas kontinuerligt framåt och det besparas mycket tid. 
\item  Avsätt mycket tid för databasen och integrationen med utvecklingsmiljö. 
\item  Sätt upp tydliga grund rules och ta nödvändiga åtgärder om det skulle behövas.        
\end{enumerate}

\section{Slutsatser}
Brief conclusion: What did you accomplish? What made you succeed or not? 

Projektgruppen lyckades bygga en fungerande rapporteringsverktyg som tillåter användare skicka in hållbarhetsdata.
\textbf{(bruhhh har vi verkligen gjort det???? ändrish sen utifrån slutresultatet)}En sammanställning av hållbarhetsdata kunde göras för att få statistik av en administratör från adminsidan på rapporteringsverktyget. Projektgruppen har tillämpat och lärt sig använda det agila arbetssättet samtidigt som vi har utifrån kundens uppdraget hjälpt till att utveckla ett  rapporteringsverktyg som kommer kunna användas till ett gott syfte för stadens hållbarhetsutveckling. Arbetet var utmanade men samtidigt väldigt lärorikt och kul. 

\section{Referenser}
[1] Sharp. H, Preece, J., \& Rogers, Y. (2019). \textit{Interaction Design - Beyond Human/Computer Interaction}. John Wiley \& Sons. Hoboken, New Jersey, United States. \\

\noindent
[2] Kniberg \& H. Skarin, M. (2010), Kanban And Scrum Making Most of Both, C4Media Inc., U.S. \\\\
\noindent
[3] Matz, K. (2013). Designing Usable Apps, An Agile Approach to User Experience Design. Hämtad 2021-
10–10 från https://canvas.education.lu.se/courses/13397/files/
1703747?wrap=1 \\\\
\noindent
[4] Eriksson, M. \& Lilliesköld J. (2005), Handbok f ̈or mindre projekt, Liber, Solna, Sweden, pp. 42–46.
\bibliographystyle{alpha}


\end{document}
