\documentclass{article}
\usepackage[utf8]{inputenc}
\usepackage{fancyhdr}
\usepackage{graphicx}
\usepackage{geometry}
\usepackage{float}

% ---- Commands ------- %

\newcommand{\documentVersion}[1]{
    \medskip
}
\newcommand{\documentTitle}[1]{
    \centerline{\rule{13cm}{0.4pt}}
    \bigskip \bigskip
    \LARGE \textbf{Frågor inför handledarmöte} \\
    \bigskip
    \LARGE {#1} \\
    \bigskip \bigskip
    \centerline{\rule{13cm}{0.4pt}}
}

\newcommand{\documentAuthors}[1]{
    \LARGE Authors: {#1} \\
    \medskip    
}
\newcommand{\documentDate}[1]{
    \date {#1} 
}

% ------------------------------------------------ #

% ----- FILL THIS ----- %
\title {
    % Full name - SHORTNAME
    \documentTitle {Helsingborg Event and Convention Bureau}
    
    % Format: YYYY-MM-DD
    \documentDate {}
}

\begin{document}

\maketitle
\thispagestyle{empty}

\newpage




\newpage

\section{Frågor v.37}

 %-- LÄGG TILL ERA FRÅGOR -- %
 
\begin{itemize}
    \item Hur tar vi fram kvalitetskraven om prickäkerhet, kapacitet, prestanda?
        \begin{description}
            \item[Svar :]
        \end{description}
    \\
     \item Kan man använda virtual windows för att skapa design förslag eller vad har de egentligen för mening?
        \begin{description}
            \item[Svar :]
        \end{description}
    \\
    \item Fråga 3
        \begin{description}
            \item[Svar :]
        \end{description}  
    \\
     \item Fråga 4
        \begin{description}
            \item[Svar :]
        \end{description}
    \\
\end{itemize}



\section{Frågor v.38}

 %-- LÄGG TILL ERA FRÅGOR -- %
 
\begin{itemize}
    \item Hur tar vi fram kvalitetskraven om prickäkerhet, kapacitet, prestanda?
        \begin{description}
            \item[Svar :]
        \end{description}
    \\
     \item Kan man använda virtual windows för att skapa design förslag eller vad har de egentligen för mening?
        \begin{description}
            \item[Svar :]
        \end{description}
    \\
    \item Fråga 3
        \begin{description}
            \item[Svar :]
        \end{description}  
    \\
     \item Fråga 4
        \begin{description}
            \item[Svar :]
        \end{description}
    \\
\end{itemize}

\end{document}