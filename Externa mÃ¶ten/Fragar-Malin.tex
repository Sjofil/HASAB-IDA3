\documentclass{article}
\usepackage[utf8]{inputenc}
\usepackage{fancyhdr}
\usepackage{graphicx}
\usepackage{geometry}
\usepackage{float}

% ---- Commands ------- %

\newcommand{\documentVersion}[1]{
    \medskip
}
\newcommand{\documentTitle}[1]{
    \centerline{\rule{13cm}{0.4pt}}
    \bigskip \bigskip
    \LARGE \textbf{Frågor inför avsämningsmöte} \\
    \bigskip
    \LARGE {#1} \\
    \bigskip \bigskip
    \centerline{\rule{13cm}{0.4pt}}
}

\newcommand{\documentAuthors}[1]{
    \LARGE Authors: {#1} \\
    \medskip    
}
\newcommand{\documentDate}[1]{
    \date {#1} 
}

% ------------------------------------------------ #

% ----- FILL THIS ----- %
\title {
    % Full name - SHORTNAME
    \documentTitle {Helsingborg Event and Convention Bureau}
    
    % Format: YYYY-MM-DD
    \documentDate {}
}

\begin{document}

\maketitle
\thispagestyle{empty}

\newpage




\newpage

\section{Frågor}

 %-- LÄGG TILL ERA FRÅGOR -- %
 
\begin{itemize}
    \item Hur ser ut data till GDSM ut?
        \begin{description}
            \item[Svar :]
        \end{description}
    \\
     \item Finns det ett tidskrav för när statistiken ska lämnas in till GDSM? Isåfall ska statistiken sammanställas automatiskt eller är det något du vill bestämma när det ska ske?
        \begin{description}
            \item[Svar :]
        \end{description}
    \\
    \item Dessa små faktarutor du nämnde förra gången. Svårt för oss att veta vad dessa ska innehålla. Vi skulle behöva ha de från dig.
        \begin{description}
            \item[Svar :]
        \end{description}  
    \\
     \item Samma sak gällande översättning av GDSM:s frågor. Vi behöver översättning. 
        \begin{description}
            \item[Svar :]
        \end{description}
    \\
    \item Vem sköter utskick av mail? (Är det något ni hade tänkt göra manuellt eller ska detta ske av systemet?)
        \begin{description}
            \item[Svar :]
        \end{description}
    \\
    \item Vem sköter utskick av mail? Vi har två förslag
    \item[--] Maillista i programmet som är strukturerad utifrån vilka brancher företagen tillhör. Programmet skickar sedan ut mail med länkar till respektive bransch.
    \item[--] - Programmet genererar koder som ni skickar ut i mail. Användaren väljer själv branch.
        \begin{description}
            \item[Svar :]
        \end{description}
    \\
    \item Design och Färger på hemsidan.
    \item[--]Har helsingborg några specifika färger som ni vill använda?
    \item[-- ]Någon logga ni vill använda er av, ska denna alltid synas isåfall?
    \item[--] Vi har en prototyp vi vill visa.
        \begin{description}
            \item[Svar :]
        \end{description}
    \\
    \item Hur högt prioriterat att lägga till frågor?
        \begin{description}
            \item[Svar :]
        \end{description}
    \\
    \item Någon kostandasgräns på hur mycket en webbserver får kosta i månaden?
        \begin{description}
            \item[Svar :]
        \end{description}
    \\
\end{itemize}

\end{document}