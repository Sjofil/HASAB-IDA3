Vi har lärt oss att målbild är viktigt. Att det är viktigt att uttala vart man är på väg, annars kan man så lätt missförstå varandra.                                                               

Hur man kan göra en övergripande bild av planeringen. 

Vi har fått tillgång till verktyg - t ex Scrum. Det tror vi kommer hjälpa oss mycket. Rent konkret. Också bra inför framtiden att man vet hur det fungerar. 

Bra input av Kirre när vi presenterade. Hon gick igenom alla delar och vi blev lite tagna på sängen så det var en rättvis bild av var vi är i projektet som hon kommenterade.

Verktyg, lärt oss prata om projektet, börjat lära oss att estimera tid. 


1. Var är vi i fasen?
Rollsökande, även om vi har definierat roller. 

2.ge konkreta och genomförbara förslag på hur ni i gruppen kan förebygga de fem "dysfunctions". Lägg störst vikt vid de som ligger längst ned i pyramiden.

Absence of trust - hur ska vi förebygga. Vi är alla vuxna, så vi kan ta en diskussion. Kommunicera med varandra. 

Fear of conflict - Kräv en debatt! Kan vara svårt att veta sin roll - bra att faktiskt göra rollfördelningen. OM folk inte vet vad de ska göra så kan det leda till passivitet vilket kan leda till konflikter. Ground rules.

Lack of commitment - lös med rollfördelning och ansvarsfördelning. Groud rules. 

Avoidance of Accountability - Scrumboarden kan hjälpa oss att se att man faktiskt har sitt namn på något.

3. Ground rules: 
	A. Utnyttja delgrupperna. Hjälpa varandra i delgrupperna i första hand. Två möten i veckan - ett på distans (fredag) och ett på plats (måndag). Ta alltid det som är högst upp. Tar man en uppgift tar man ansvar för att den ska slutföras. Feedback på fredagar i grupp. Alla tar upp vad man har gjort under veckan.
 	B. Vi har kommunikation på discord. Reagera på poster så att vi vet att alla tagit del av informationen. Med så pass många möten borde vi kunna få ut rätt information i tid. Kolla discord några gångre per dag mellan 8-16.
	C. Meddela i förväg om man får förhinder. Hur beter man sig? Lyssna på det som övriga deltagare säger. Var delaktig. 
	

Filip Anteckningar:

4. Fördela ansvarsområden

    A. 
        - Kravprocessbok 
        - Processbok
        - Kravdokument
        - Scrumboard
        - Mail
        - Möten/Dagordning
    
    B.
        - Kravprocessbok: Insatt i kravarbetet, övergripande kunskap om projektet, nogrann, Latex kunnig.
        - Processboken: Insatt i övningar, övergripande kunskap om projektet, nogrann, Latex kunnig.
        - Kravdokument: Insatt i kravhantering och hur krav utformas, detaljkunnig, mycket insatt i vertyget oss dess funktioner, ha koll på semantik.
        - Scrumboard: Administrativ, hålla koll på alla uppgifter som finns att göra i projektet, person med bra övergripande koll, strukturerad.
        - Mail: Socialt kunnig, administrativ.
        - Möten/Dagordning: Ledarroll, bra koll på vad som är viktigt i stunden.
        
    C.
        - I stort sätt samma på alla ansvarsområden. Kommunikativ och aktiv. Dela med sig av nyheter/ändringar inom respektive ansvarsområde så att alla har möjlighet att ta del av informationen. 
        
        - Kravprocessbok: Vara uppdaterad på hur kravhanteringen går till.
        - Processboken: Vara delaktig på möten och workshops. 
        - Kravdokument: Håll koll på produkten och hur den ska fungera.
        - Scrumboard: Vara uppdaterad kring projektet och vad som är akutellt att jobba på. 
        - Mail: Responsiv och ta ansvar att folk får viktigt information
        - Möten/Dagsordning: Håll koll på vad som är viktigt i nuläget.
        
5. 
    Kunskap från Belbin-testet kan göra det enklare att dela ut roller eftersom det ger en inblick i hur varje person fungerar. Medlammar i gruppen får en bättre förståelse över hur personer i gruppen fungerar i vissa situationer och kan lättare komma fram till bra lösningar som passar alla. 
    
    I projektgruppen kan det tillexempel handla om att under övningar bättre kunna få fram folks tankar. Vet man att någon har som preferens att inte prata kan man fråga personen mer frågor..? Ett annat exempel kan vara vid om projektledaren är perfektionst och har svårt att deligera ut arbetet, att veta detta kan göra att andra medlammar vågar säga ifrån och erbjuda hjälp...? Kommer ni på något mer skriv gärna in. 
    